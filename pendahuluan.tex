
\section{Pelanggaran dan Sanksi}
    
\par Segala bentuk pelanggaran tata tertib maupun tindakan kecurangan akademik; seperti melihat catatan atau pekerjaan orang lain, kerja sama dengan peserta lain atau mahasiswa di luar ruangan, dan menggantikan atau digantikan oleh mahasiswa lain pada saat ujian; sesuai ketentuan/ketetapan yang ada dapat dikenakan sanksi mulai dari sanksi akademik berupa: 
\begin{itemize}[noitemsep]
    \item Pembatalan nilai (pemberian nilai E)
    \item Pembatalan studi satu semester
    \item Skorsing
    \item Dikeluarkan (pemberhentian sebagai mahasiswa) dari FTUI.
\end{itemize}
Bila diperlukan, dapat melalui sidang pemeriksaan Panitia Penyelesaian Pelanggaran
Tata Tertib (P3T2).
    \section{Disclaimer}
\begin{enumerate}[label=\Alph*.]
    \item Diktat ini \textbf{BUKAN} merupakan kisi-kisi atau hal sejenisnya yang merujuk pada soal ujian yang akan diberikan
    \item Diktat ini dibuat dan dikerjakan pembahasannya oleh mahasiswa dengan bekal ilmu yang sudah didapatnya dengan tujuan mematangkan konsep dasar dalam menjawab soal, cara pengerjaan soal-soal mungkin berbeda dengan yang diinginkan dosen Anda, gunakanlah cara yang dianjurkan dosen Anda jika ada.
\end{enumerate}


\begin{flushleft}
    \textbf{Contact Person:} \\
    Natano Juditya Sihan (Line Id: natanojs, WA: 081212361505) \\
    Kevin Imanuel Hutagaol (Line ID: kevin\_imanuel62, WA: 081345519179)
\end{flushleft}
\vspace{10pt}
\begin{center} 
    BIDANG AKADEMIS DAN KEPROFESIAN \\
    IKATAN MAHASISWA ELEKTRO \\
    FAKULTAS TEKNIK \\
    UNIVERSITAS INDONESIA 
\end{center}
\begin{flushright} 
    \textbf{\textit{ime.eng.ui.ac.id/akademis}}
\end{flushright}

\newpage
    \section{Kata Pengantar}
Puji dan syukur ke hadirat Tuhan Yang Maha Esa atas segala rahmat dan karunia-Nya sehingga diktat \namamatkul ini dapat diselesaikan dengan baik. Ucapan terima kasih juga kami sampaikan kepada para dosen, asisten dosen dan laboratorium, serta teman-teman yang telah berkontribusi dalam proses pengerjaan dan pengecekan diktat ini.
\vspace{12pt}

Kami dari pihak Akpro IME FTUI berharap agar diktat \namamatkul ini dapat membantu mahasiswa dalam mempersiapkan diri untuk menghadapi Ujian Tengah Semester Genap ini. Semoga diktat \namamatkul ini bisa menambah pengetahuan dan keterampilan mahasiswa sehingga mampu menjawab soal-soal ujian yang akan dihadapi dengan maksimal. 
\vspace{12pt}

Adapun karena keterbatasan pengetahuan maupun pengalaman kami, kami menyadari masih terdapat kekurangan dalam penyusunan diktat ini yang perlu diperbaiki. Oleh karena itu, kritik dan saran sangat kami harapkan sehingga dapat dijadikan bahan evaluasi dan perbaikan untuk diktat yang lebih baik lagi. Kami juga memohon maaf apabila ada kesalahan dan kekurangan dalam penyusunan diktat ini.
\vspace{12pt}

\textbf{Akpro IME FTUI menegaskan bahwa mempelajari diktat ini tidak menjamin kelulusan mahasiswa pada mata kuliah yang berkaitan,} namun kami berharap diktat ini dapat membantu mahasiswa untuk belajar dan memahami lebih lanjut mata kuliah yang akan diujikan saat UTS ini. \textbf{Diktat ini hanya berfungsi sebagai suplemen} sehingga pada ujian nanti nilai mahasiswa tidak ditentukan oleh diktat ini, namun ditentukan oleh usaha masing-masing individu untuk mendapatkan hasil yang memuaskan.\\
\afterpage{
    \pagenumbering{arabic}
    \setcounter{page}{1}
}
%--------------------------------------------------------------------------------------
